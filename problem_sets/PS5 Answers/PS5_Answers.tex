\documentclass[12pt,letterpaper]{article}
\usepackage{graphicx,textcomp}
\usepackage{natbib}
\usepackage{setspace}
\usepackage{fullpage}
\usepackage{color}
\usepackage[reqno]{amsmath}
\usepackage{amsthm}
\usepackage{fancyvrb}
\usepackage{amssymb,enumerate}
\usepackage[all]{xy}
\usepackage{endnotes}
\usepackage{lscape}
\newtheorem{com}{Comment}
\usepackage{float}
\usepackage{hyperref}
\newtheorem{lem} {Lemma}
\newtheorem{prop}{Proposition}
\newtheorem{thm}{Theorem}
\newtheorem{defn}{Definition}
\newtheorem{cor}{Corollary}
\newtheorem{obs}{Observation}
\usepackage[compact]{titlesec}
\usepackage{dcolumn}
\usepackage{tikz}
\usetikzlibrary{arrows}
\usepackage{multirow}
\usepackage{xcolor}
\newcolumntype{.}{D{.}{.}{-1}}
\newcolumntype{d}[1]{D{.}{.}{#1}}
\definecolor{light-gray}{gray}{0.65}
\usepackage{url}
\usepackage{listings}
\usepackage{color}

\definecolor{codegreen}{rgb}{0,0.6,0}
\definecolor{codegray}{rgb}{0.5,0.5,0.5}
\definecolor{codepurple}{rgb}{0.58,0,0.82}
\definecolor{backcolour}{rgb}{0.95,0.95,0.92}

\lstdefinestyle{mystyle}{
	backgroundcolor=\color{backcolour},   
	commentstyle=\color{codegreen},
	keywordstyle=\color{magenta},
	numberstyle=\tiny\color{codegray},
	stringstyle=\color{codepurple},
	basicstyle=\footnotesize,
	breakatwhitespace=false,         
	breaklines=true,                 
	captionpos=b,                    
	keepspaces=true,                 
	numbers=left,                    
	numbersep=5pt,                  
	showspaces=false,                
	showstringspaces=false,
	showtabs=false,                  
	tabsize=2
}
\lstset{style=mystyle}
\newcommand{\Sref}[1]{Section~\ref{#1}}
\newtheorem{hyp}{Hypothesis}

\title{Problem Set 5}
\date{Due: March 4, 2020}
\author{QTM 200: Applied Regression Analysis}

\begin{document}
	\maketitle
	
	\section*{Instructions}
	\begin{itemize}
		\item Please show your work! You may lose points by simply writing in the answer. If the problem requires you to execute commands in \texttt{R}, please include the code you used to get your answers. Please also include the \texttt{.R} file that contains your code. If you are not sure if work needs to be shown for a particular problem, please ask.
		\item Your homework should be submitted electronically on the course GitHub page in \texttt{.pdf} form.
		\item This problem set is due at the beginning of class on Wednesday, March 4, 2020. No late assignments will be accepted.
		\item Total available points for this homework is 100.
	\end{itemize}
	
		\vspace{.5cm}
	
\noindent  Using the \texttt{teengamb} dataset, fit a model with \texttt{gamble} as the response and the other variables as predictors. 

\vspace{.5cm}
\lstinputlisting[language=R, firstline=41, lastline=43]{PS5.R}  
\vspace{.5cm}
Answer the following questions:
\vspace{.5cm}
\begin{enumerate}[(a)]
	 \item Check the constant variance assumption for the errors by plotting the residuals versus the fitted values. \newpage
\vspace{1cm}
\lstinputlisting[language=R, firstline=46, lastline=46]{PS5_Answers.R} 
\begin{figure} [H]
	\includegraphics[width=\linewidth]{residual_vs_fitted.jpeg}
\end{figure}
\vspace{1cm}
\noindent Overall, the plot looks good and the constant variance assumption seems to be satisfied. There is a larger amount of variance in the middle of the plot around 30 on the x-axis compared to the edges of the plot but it does not seem drastic.
\vspace{3cm}

	\item Check the normality assumption with a Q-Q plot of the studentized residuals. \vspace{1cm}
\lstinputlisting[language=R, firstline=49, lastline=49]{PS5_Answers.R} 
\begin{figure} [H]
	\includegraphics[width=\linewidth]{qq_plot.jpeg}
\end{figure}
\vspace{1cm}
\noindent The normality assumption appears to be satisfied because the points fall along a straight line on average with the exception of 24, 36, and 39.
\vspace{3cm}	
	\item Check for large leverage points by plotting the $h$ values. \vspace{1cm}
\lstinputlisting[language=R, firstline=52, lastline=55]{PS5_Answers.R} 
\begin{figure} [H]
	\includegraphics[width=\linewidth]{hvalues.jpeg}
\end{figure}
\vspace{1cm}
\noindent There are two large leverage points: 35 and 42.
\vspace{3cm}
	\item Check for outliers by running an \texttt{outlierTest}. \vspace{1cm}
\lstinputlisting[language=R, firstline=58, lastline=59]{PS5_Answers.R} 
\vspace{1cm}
\noindent The adjusted p-value for the largest error hat*i is larger than 0.05, we conclude that this model does not have any extreme residuals.
\vspace{3cm}	
	\item Check for influential points by creating a "Bubble plot" with the hat-values and studentized residuals.
\lstinputlisting[language=R, firstline=62, lastline=68]{PS5_Answers.R} 
\begin{figure} [H]
	\includegraphics[width=\linewidth]{bubbleplot.jpeg}
\end{figure}
\vspace{1cm}
\noindent The influential points are 24, 31, 33, 35, 39, and 42.

\end{enumerate}

\end{document}

pdf latex PS5_Answers